\documentclass{article}
\usepackage[utf8]{inputenc}
\usepackage[spanish]{babel}
\usepackage[margin=1in]{geometry}
\usepackage{amsmath}
\usepackage{graphicx}
\usepackage{booktabs}
\usepackage{longtable}
\usepackage{titling}
\usepackage{float}
\usepackage{fancyhdr}
\usepackage{xcolor}

\definecolor{burgundy}{RGB}{128, 0, 32}

\pagestyle{fancy}
\fancyhf{} % Limpiar encabezado y pie
\renewcommand{\headrulewidth}{0pt} % Sin línea en el encabezado
\renewcommand{\footrulewidth}{0pt} % Sin línea predeterminada en el pie

\fancyfoot[C]{%
	\color{burgundy}\rule{\textwidth}{4pt}\\ % Franja de 4pt de alto
	\vspace{2mm} % Espacio después de la franja
	\color{black} % Texto normal después de la franja
	\thepage % Número de página
}

\title{
	\textbf{Análisis de Datos Faltantes, Atípicos y Cardinalidad en Conjuntos de Datos} \\ 
	\vspace{0.3cm}
	\large \textbf{Análisis de la Encuesta Nacional sobre la Dinámica de las Relaciones en los Hogares (ENDIREH) 2021}
}
\author{
	Yhael Salvador Pérez Balderas \\ 
	\small Universidad Autónoma de Querétaro \\ 
	\small Aprendizaje Automático
}
\date{Febrero 2026}

\begin{document}

\maketitle

\section{Introducción}

El análisis de datos es una etapa crítica en el desarrollo de modelos de aprendizaje automático, ya que la calidad de los datos determina directamente el rendimiento de los algoritmos. En este trabajo se aborda el conjunto de datos ENDIREH 2021 para identificar y tratar tres problemas comunes en bases de datos reales: datos faltantes, valores atípicos y alta cardinalidad en variables categóricas.

\section{Marco Teórico}

\subsection{Datos faltantes}
Los datos faltantes son valores ausentes en observaciones específicas. Se clasifican en:

\begin{itemize}
	\item \textbf{MCAR} (Missing Completely at Random): Valores faltantes por azar, sin relación con otras variables.
	\item \textbf{MAR} (Missing at Random): Ausencia dependiente de otras variables observadas.
	\item \textbf{MNAR} (Missing Not at Random): Datos faltantes relacionados con el valor mismo.
\end{itemize}

\subsection{Métodos de imputación}
La imputación es el proceso de reemplazar valores faltantes con estimaciones:

\begin{itemize}
	\item \textbf{Imputación por media}: Promedio de la variable. Para distribuciones simétricas.
	\item \textbf{Imputación por mediana}: Para distribuciones sesgadas o con outliers.
	\item \textbf{Imputación por moda}: Para variables categóricas.
	\item \textbf{Imputación por regresión}: Usa modelo de regresión lineal basado en variables correlacionadas.
	\item \textbf{Imputación KNN}: Promedio de los k-vecinos más cercanos.
\end{itemize}

\subsection{Datos atípicos}
Los datos atípicos se detectan con el Rango Intercuartil (IQR): valores fuera de $[Q1 - 1.5 \times IQR,\ Q3 + 1.5 \times IQR]$.

\subsection{Riesgos del ruido en los datos}
El ruido puede causar: distorsión de patrones, aumento de varianza, y reducción de precisión.

\subsection{Cardinalidad}
Número de valores únicos en una variable. Alta cardinalidad complica el modelado.

\subsection{Normalización}
La normalización es el proceso de escalar los valores de las variables a un rango específico, comúnmente $[0,1]$. Se utiliza para evitar que variables con magnitudes diferentes dominen el análisis y para mejorar el rendimiento de algoritmos de aprendizaje automático. En este análisis se aplicó la normalización Min-Max mediante la fórmula:
\[
X_{\text{norm}} = \frac{X - X_{\text{min}}}{X_{\text{max}} - X_{\text{min}}}
\]
donde $X_{\text{min}}$ y $X_{\text{max}}$ son los valores mínimo y máximo de la variable, respectivamente.

\subsection{Distancias}
\begin{itemize}
	\item \textbf{Euclidiana}: $d(x,y) = \sqrt{\sum (x_i - y_i)^2}$ \\
	Recomendada para espacios continuos sin outliers, ya que considera la distancia directa entre puntos.
	
	\item \textbf{Manhattan}: $d(x,y) = \sum |x_i - y_i|$ \\
	Preferible en presencia de outliers o en espacios con restricciones de movimiento (ej: cuadrículas urbanas), ya que es menos sensible a valores extremos.
	
	\item \textbf{Chebyshev}: $d(x,y) = \max |x_i - y_i|$ \\
	Útil cuando el costo está determinado por la dimensión más desfavorable (ej: tiempo de procesamiento en sistemas paralelos).
\end{itemize}

\section{Materiales y Métodos}

\subsection{Base de datos ENDIREH 2021}
La Encuesta Nacional sobre la Dinámica de las Relaciones en los Hogares (ENDIREH) 2021 es un programa estadístico del INEGI que genera información sobre la organización de los hogares mexicanos, las relaciones de poder al interior de la familia y la prevalencia de violencia contra las mujeres. El archivo \texttt{TSDem.csv} corresponde al módulo de Características Sociodemográficas, el cual registra información sobre edad, género, nivel educativo, parentesco y actividad económica de los residentes de la vivienda. El conjunto analizado contiene \textbf{432,746 registros} y \textbf{37 variables} originales.

\subsection{Selección de variables y eliminación de ruido}
Para garantizar la validez del análisis, se excluyeron dos tipos de variables:

\begin{enumerate}
	\item \textbf{Ruido estructural}: variables técnicas que no representan características humanas:
	\begin{itemize}
		\item \texttt{N\_REN}: Número de renglón (01--30). Identificador secuencial del cuestionario, no una variable sustantiva.
	\end{itemize}
	
	\item \textbf{Variables con $>50\%$ de valores nulos}: descartadas por no aportar información estadísticamente significativa (Tabla~\ref{tab:variables-descartadas}).
\end{enumerate}

Esta depuración evita la introducción de ruido que distorsionaría la identificación de patrones reales de datos faltantes (MCAR, MAR, MNAR).

\subsection{Variables analizadas}
Las 14 variables seleccionadas para el análisis (con $<50\%$ de valores nulos y relevancia sustantiva) se presentan en la Tabla~\ref{tab:variables-analizadas}:

\begin{table}[h]
	\centering
	\caption{Variables analizadas del archivo TSDem}
	\label{tab:variables-analizadas}
	\small
	\begin{tabular}{llp{7cm}}
		\hline
		\textbf{Variable} & \textbf{Tipo} & \textbf{Descripción} \\
		\hline
		\texttt{EDAD} & Numérico & Edad en años cumplidos (00--96, 97=97+ años) \\
		\texttt{SEXO} & Numérico & Género (1=Hombre, 2=Mujer) \\
		\texttt{NIV} & Numérico & Nivel educativo (00=Ninguno, 01=Preescolar, ..., 11=Posgrado) \\
		\texttt{GRA} & Numérico & Grado escolar aprobado (0--8) \\
		\texttt{PAREN} & Numérico & Parentesco con jefe/a del hogar (01=Jefa/Jefe, ..., 11=Empleada(o) doméstica(o)) \\
		\texttt{CVE\_ENT} & Numérico & Clave de entidad federativa (01--32) \\
		\texttt{CVE\_MUN} & Numérico & Clave de municipio (001--999) \\
		\texttt{P2\_9} & Numérico & Asiste actualmente a la escuela (1=Sí, 2=No) \\
		\texttt{P2\_10} & Numérico & Autoidentificación indígena (1=Sí, 2=Sí en parte, 3=No) \\
		\texttt{P2\_11} & Numérico & Habla dialecto/lengua indígena (1=Sí, 2=No) \\
		\texttt{P2\_13} & Numérico & Trabajó la semana pasada (1=Sí, 2=No) \\
		\texttt{P2\_16} & Numérico & Estado civil (01=Unión libre, ..., 06=Soltera(o)) \\
		\hline
	\end{tabular}
\end{table}

\subsection{Variables descartadas}
Se descartaron \textbf{8 variables} con más del 50\% de valores nulos (Tabla~\ref{tab:variables-descartadas}). Los porcentajes exactos se calcularon mediante \texttt{df.isnull().mean() * 100}:

\begin{table}[h]
	\centering
	\caption{Variables descartadas por exceso de valores faltantes ($>50\%$)}
	\label{tab:variables-descartadas}
	\small
	\begin{tabular}{llp{5cm}}
		\hline
		\textbf{Variable} & \textbf{\% Nulos} & \textbf{Motivo de descarte} \\
		\hline
		\texttt{P2\_12} & 93.63\% & Habla español (no aplicable en contextos donde es lengua dominante) \\
		\texttt{REN\_MUJ\_EL} & 74.55\% & Renglón de mujer elegida (solo aplica a submuestra de 15+ años) \\
		\texttt{REN\_INF\_AD} & 71.66\% & Renglón de informante adecuado (solo aplica a submuestra específica) \\
		\texttt{P2\_8} & 63.66\% & Sabe leer/escribir (no aplicable a menores de edad) \\
		\texttt{P2\_14} & 62.66\% & Tipo de trabajo (no aplicable a población económicamente inactiva) \\
		\texttt{CODIGO} & 60.81\% & Código de mujer seleccionada (solo aplica a submuestra) \\
		\texttt{COD\_M15} & 60.81\% & Código para mujeres de 15+ años (solo aplica a submuestra) \\
		\texttt{P2\_15} & 53.37\% & Relación laboral (no aplicable a población sin actividad económica) \\
		\hline
	\end{tabular}
\end{table}

\subsection{Metodología aplicada}
El análisis siguió el siguiente flujo:

\begin{enumerate}
	\item Carga del conjunto de datos con codificación \texttt{latin-1}.
	\item Cálculo del porcentaje de valores nulos por variable mediante \texttt{df.isnull().mean() * 100}.
	\item Descarte de 7 variables con $>50\%$ nulos y eliminación de ruido estructural (\texttt{N\_REN}).
	\item Detección de datos atípicos mediante el método del Rango Intercuartil (IQR), calculando los percentiles 25 y 75 para cada variable numérica y considerando como outliers los valores fuera de $[Q1 - 1.5 \times IQR,\ Q3 + 1.5 \times IQR]$.
	\item Generación de boxplots iniciales para visualizar distribución y detectar outliers (Figura~\ref{fig:boxplot_inicial}).
	\item Selección específica de método de imputación por variable:
	\begin{itemize}
		\item \textbf{EDAD}: Imputación por \textit{media} (distribución simétrica, sesgo $<$ 0.1).
		\item \textbf{NIV}: Imputación por \textit{moda}.
		\item \textbf{Resto de variables}: Imputación por \textit{mediana}.
	\end{itemize}
	\item Aplicación de imputación mediante \texttt{SimpleImputer} (media/mediana/moda).
	\item Normalización con \texttt{MinMaxScaler} para escalar las variables imputadas al rango $[0,1]$, utilizando la implementación de \texttt{scikit-learn}.
	\item Generación de boxplots (Figura~\ref{fig:boxplot_inicial}) incluyendo comparación específica EDAD-NIV (Figura~\ref{fig:boxplot_edad_niv}).
\end{enumerate}

Todas las operaciones se realizaron en Python 3.13 usando las librerías \texttt{pandas}, \texttt{numpy}, \texttt{matplotlib}, \texttt{seaborn} y \texttt{scikit-learn}. El código completo se encuentra en \texttt{src/analisis\_endireh.py}.



\section{Resultados y Discusión}

\subsection{Resultados de imputación}
La Figura~\ref{fig:boxplot_inicial} muestra la distribución inicial de las variables antes de la imputación. La imputación por media preservó la simetría de \texttt{EDAD}. La imputación por moda aplicada a \texttt{NIV} permitió manejar los valores faltantes de forma eficiente.

\subsection{Análisis de datos faltantes}
El análisis identificó 8 variables con más del 50\% de valores nulos (Tabla~\ref{tab:variables-descartadas}), siendo \texttt{P2\_12} la más afectada (93.63\%). Estas variables corresponden principalmente a preguntas aplicables solo a submuestras específicas (ej: mujeres de 15+ años), lo que explica su alta tasa de no respuesta. La distribución inicial de las variables analizadas mostró sesgo positivo en \texttt{NIV} (nivel educativo) y simetría en \texttt{EDAD}.

\subsection{Efecto de la imputación}
Se aplicaron métodos específicos según la naturaleza de cada variable. Para \texttt{EDAD}, se utilizó imputación por media debido a su distribución simétrica (sesgo menor a 0.1). En el caso de \texttt{NIV}, se aplicó imputación por moda, ya que esta variable es ordinal y no presenta correlación significativa con otras variables, por lo que el valor más frecuente resultó el método más apropiado. Para el resto de las variables, se empleó imputación por mediana debido a sus distribuciones sesgadas.

\begin{figure}[H]
	\centering
	\includegraphics[width=0.9\textwidth]{../figures/boxplot_inicial.png}
	\caption{Distribución antes de imputación}
	\label{fig:boxplot_inicial}
\end{figure}

\begin{figure}[H]
	\centering
	\includegraphics[width=0.9\textwidth]{../figures/boxplot_edad_niv.png}
	\caption{Comparación EDAD (media) y NIV (moda)}
	\label{fig:boxplot_edad_niv}
\end{figure}


\subsection{Resumen estadístico}
La Tabla~\ref{tab:resumen} presenta el resumen estadistico post-imputacion.

\begin{table}[H]
	\centering
	\begin{tabular}{|l|c|c|c|c|c|c|c|}
		\hline
		Variable & Count & Mean & Std & Min & 25\% & 50\% & Max \\
		\hline
		ID\_VIV & 432746 & 1671565.43 & 919294.58 & 100003.01 & 903905.02 & 1661346.19 & 3260787.18 \\
		UPM & 432746 & 1671565.37 & 919294.58 & 100003.00 & 903905.00 & 1661346.00 & 3260787.00 \\
		VIV\_SEL & 432746 & 6.89 & 5.69 & 1.00 & 3.00 & 5.00 & 24.00 \\
		CVE\_ENT & 432746 & 16.53 & 9.17 & 1.00 & 9.00 & 16.00 & 32.00 \\
		CVE\_MUN & 432746 & 39.02 & 62.01 & 1.00 & 7.00 & 19.00 & 570.00 \\
		HOGAR & 432746 & 1.03 & 0.21 & 1.00 & 1.00 & 1.00 & 6.00 \\
		N\_REN & 432746 & 2.77 & 1.79 & 1.00 & 1.00 & 2.00 & 24.00 \\
		PAREN & 432746 & 2.63 & 1.72 & 1.00 & 1.00 & 3.00 & 11.00 \\
		SEXO & 432746 & 1.51 & 0.50 & 1.00 & 1.00 & 2.00 & 2.00 \\
		EDAD & 432746 & 32.98 & 21.42 & 0.00 & 15.00 & 30.00 & 99.00 \\
		P2\_5 & 432746 & 54.32 & 46.90 & 1.00 & 2.00 & 96.00 & 98.00 \\
		P2\_6 & 432746 & 65.87 & 44.44 & 1.00 & 1.00 & 96.00 & 98.00 \\
		NIV & 432746 & 3.84 & 2.89 & 0.00 & 2.00 & 3.00 & 11.00 \\
		GRA & 416098 & 3.12 & 1.62 & 0.00 & 2.00 & 3.00 & 9.00 \\
		P2\_9 & 416098 & 1.71 & 0.45 & 1.00 & 1.00 & 2.00 & 2.00 \\
		P2\_10 & 416098 & 2.53 & 1.14 & 1.00 & 1.00 & 3.00 & 8.00 \\
		P2\_11 & 416098 & 1.93 & 0.25 & 1.00 & 2.00 & 2.00 & 2.00 \\
		P2\_13 & 351278 & 1.46 & 0.50 & 1.00 & 1.00 & 1.00 & 2.00 \\
		P2\_16 & 351278 & 4.35 & 1.86 & 1.00 & 3.00 & 5.00 & 6.00 \\
		FAC\_VIV & 432746 & 295.70 & 252.31 & 12.00 & 132.00 & 226.00 & 2826.00 \\
		FAC\_MUJ & 432746 & 116.75 & 326.98 & 0.00 & 0.00 & 0.00 & 8836.00 \\
		ESTRATO & 432746 & 2.14 & 0.82 & 1.00 & 2.00 & 2.00 & 4.00 \\
		EST\_DIS & 432746 & 298.81 & 176.60 & 1.00 & 154.00 & 300.00 & 617.00 \\
		UPM\_DIS & 432746 & 9276.42 & 5065.16 & 1.00 & 4953.00 & 9192.00 & 17795.00 \\
		\hline
	\end{tabular}
	\caption{Resumen estad\'istico de variables num\'ericas}
	\label{tab:resumen}
\end{table}

\subsection{Implicaciones}
Los resultados confirman que los datos faltantes en variables demográficas como \texttt{EDAD} siguen patrones MCAR/MAR y son manejables con métodos simples como la imputación por media. Por otro lado, las variables sensibles como \texttt{NIV} requieren enfoques por moda debido a su naturaleza MNAR, ya que no presentan correlación significativa con otras variables y su naturaleza ordinal exige preservar valores categóricos válidos. La imputación aplicada preservó las relaciones clave entre variables, permitiendo un análisis válido de factores asociados a la violencia de género en la ENDIREH 2021.

\section{Conclusión}

La imputación por media y moda, aplicadas según la naturaleza de cada variable, permitieron completar el conjunto de datos para su análisis. Dado que los métodos seleccionados respetan la distribución original de las variables (media para distribuciones simétricas, moda para variables ordinales), se minimizó el riesgo de distorsionar las relaciones esperadas entre variables demográficas en encuestas sociales como la ENDIREH.

\section{Bibliografía}

Aceves Fernández, M. A. (2024). \textit{De 0 a 100 en Inteligencia Artificial}.

Aceves Fernández, M. A. (2026). Presentaciones. Material de clase, UAQ.

Géron, A. (2019). \textit{Hands-On Machine Learning}. O'Reilly Media.

INEGI. (2021). ENDIREH 2021. https://www.inegi.org.mx/programas/endireh/2021/

\end{document}